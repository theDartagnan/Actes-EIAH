\begin{center}
	\noindent\large\textbf{ Personnaliser l’Apprentissage dans l’Enseignement Supérieur ?\\
		Pas si simple. }
\end{center}

\begin{center}
	\begin{tabular}{@{}c@{}}
		Agathe Merceron, Professeure d'Informatique\\
		Computer Science at the Berlin University of Applied Sciences, Germany\\
		\email{merceron@bht-berlin.de}
	\end{tabular}
\end{center}

\subsection*{Résumé}
Dans l’apprentissage comme dans les autres domaines, nous avons de plus en plus de données numériques que nous pouvons donc analyser avec des algorithmes. 
Et nous faisons des découvertes intéressantes ! 
Par exemple, nous pouvons prédire si un étudiant ou une étudiante va interrompre ses études avec une exactitude de plus de 90\% dans certains cas.
Ou bien encore, nous pouvons extraire des comportements d’étudiants dans les plateformes numériques et relier certains comportements avec un plus grand succès académique. 
Comment utiliser ces ``trouvailles'' pour aider les étudiants à mieux étudier ? 
Que pensent les étudiants de ces “trouvailles” et que souhaitent-ils, que souhaitent-elles ? 
Dans cette conférence, je vais présenter quelques ``trouvailles'' que nous avons faites dans nos projets et comment nous essayons d’impliquer les étudiants pour recueillir leurs opinions et points de vue. 

\subsection*{Biographie}

\begin{wrapfigure}[12]{R}{0.25\textwidth}
	\centering\includegraphics[width=0.24\textwidth]{Content/figures/AgatheMerceron}
\end{wrapfigure}
\textbf{Agathe Merceron} est Professeure émérite d’Informatique à l’Université des
Sciences Appliquée de Berlin où  elle y a dispensé différents enseignements
tels que l’introduction à la programmation, les fondements théoriques de
l’Informatique ou l’apprentissage machine. 
Jusqu'à Mars 2022, elle était directrice des programmes d'enseignement en ligne d'Informatique et des  médias pour les Bachelors et Masters. 
Sa recherche porte actuellement sur les Environnements Informatiques pour l'Apprentissage Humain avec une attention sur la fouille de données d'apprentissage (\textit{Education Data Mining}) et sur les \textit{Learning Analytics}. 
Elle est impliquée nationalement et à l'international dans ces domaines, a été la président de comités de programme de conférences et workshops, en particulier pour les conférences internationales ``Educational Data Mining'' (EDM) et ``Learning Analytics and Knowledge'' (LAK), est éditrice de la revue  ``Journal of Educational Data Mining'' (JEDM) et est membre du comité de programme du journal ''Sciences et Technologies de l'Information et de la  Communication pour l'Éducation et la Formation'' (STICEF).





